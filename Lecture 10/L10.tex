%----------------------------------------------------------------------------------------
%    PACKAGES AND THEMES
%----------------------------------------------------------------------------------------

\documentclass[aspectratio=169,xcolor=dvipsnames]{beamer}
\usetheme{SimplePlus}

\usepackage{tikz}
\usepackage{hyperref}
\usepackage{graphicx} % Allows including images
\usepackage{booktabs} % Allows the use of \toprule, \midrule and \bottomrule in tables
\usepackage{wrapfig}
\usepackage{listings}
\usepackage[font=small,labelfont=bf]{caption}

%----------------------------------------------------------------------------------------
%    TITLE PAGE
%----------------------------------------------------------------------------------------

\title{Neural Network \& Deep Learning}
\subtitle{HI 743}

\author{Ryan Gallagher}

\institute
{
    Department of Health Informatics and Administration \\
    Zilber College of Public Health \\
    University of Wisconsin - Milwaukee% Your institution for the title page
}
\date{April 10th, 2025} % Date, can be changed to a custom date


%----------------------------------------------------------------------------------------
%    PRESENTATION SLIDES
%----------------------------------------------------------------------------------------

\begin{document}
\begin{frame}
    % Print the title page as the first slide
    \titlepage
\end{frame}

%----------------------------------------------------------------------------------------
%    Outline
%----------------------------------------------------------------------------------------

%\begin{frame}{Overview}
    % Throughout your presentation, if you choose to use \section{} and \subsection{} commands, these will automatically be printed on this slide as an overview of your presentation
%    \tableofcontents
%\end{frame}

%----------------------------------------------------------------------------------------
%    Slides
%----------------------------------------------------------------------------------------

\section{Overview}

\begin{frame}{What is Deep Learning?}
  \begin{block}{Definition}
    Deep learning is a class of machine learning techniques that use \textbf{multi-layered neural networks} to learn data representations and predictive models.
  \end{block}

  \vspace{0.5em}

  \begin{itemize}
    \item Capable of modeling \textbf{highly complex, non-linear} relationships in data.
    \item Processes raw inputs through multiple layers to learn useful features \textit{automatically}.
    \item Particularly effective for tasks involving \textbf{images}, \textbf{speech}, \textbf{text}, and \textbf{healthcare records}.
    \item Enables end-to-end learning: from raw input to final prediction.
  \end{itemize}
\end{frame}

\begin{frame}{Neural Networks}
  \begin{block}{What is a Neural Network?}
    A \textbf{neural network} is a layered mathematical model designed to approximate complex functions by combining many simple units (neurons).
  \end{block}

  \vspace{0.5em}

  \begin{itemize}
    \item Composed of \textbf{layers}:
      \begin{itemize}
        \item \textbf{Input layer}: Receives raw features (e.g., patient age, lab values).
        \item \textbf{Hidden layers}: Transform inputs using weighted combinations and activation functions.
        \item \textbf{Output layer}: Produces the final prediction (e.g., diagnosis probability).
      \end{itemize}
    \item Each unit (\textbf{neuron}) computes a weighted sum of its inputs, applies a non-linear function, and passes it forward.
    \item The network \textbf{learns} by adjusting weights to minimize prediction error, often using \textit{gradient descent}.
    \item With enough hidden layers, a neural network can approximate almost any function (\textbf{Universal Approximation Theorem}).
  \end{itemize}
\end{frame}

\begin{frame}{Neural Networks Explained}
\centering
But what is a neural network? | by 3Blue1Brown
	\begin{center}
    \href{https://www.youtube.com/watch?v=aircAruvnKk}{%
      \includegraphics[width=0.5\linewidth]{images/utube.jpg}%
    }
  \end{center}
  
\end{frame}




\end{document}